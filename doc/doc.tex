\documentclass[a4paper]{article}
\usepackage{a4}
\usepackage{iso8859-1}
\usepackage{epsfig}

\newenvironment{bfscript}[1]{%
  \begin{list}{}{\settowidth{\labelwidth}{\bf #1}
      \setlength{\leftmargin}{\labelwidth} 
      \addtolength{\leftmargin}{\labelsep}
      \parsep0.5ex plus0.2ex minus0.2ex \itemsep 0.3ex 
      \renewcommand{\makelabel}[1]{\bf ##1\hfill}}}
  {\end{list}}
\newcommand{\new}{%
        {\ooalign {\hfil \raise .09ex
         \hbox {n}\hfil \crcr\mathhexbox 20D}}}

\begin{document}
\title{Ebnf2ps --- Automatic Railroad Diagram Drawing\\
       Version 1.01}
\author{
  Peter Thiemann, Michael Walter\\
  Wilhelm-Schickard-Institut, Universit�t T�bingen\\
  Sand 13, 72076 T�bingen\\
  E-mail: \texttt{\{thiemann,walterm\}@informatik.uni-tuebingen.de} \\
  \copyright 1996, Peter Thiemann}
\maketitle

Ebnf2ps  was written to fill an urgent need of the author: to produce lots of
syntax diagrams (railroad diagrams) from a grammar in EBNF notation in short
time. Over the time the program has acquired a couple of features in order to
fulfill a variety of requirements. The current version produces highly readable
railroad diagrams from grammars in various formats: Yacc, Happy and
EBNF. \\
The program may be freely used and distributed, you are however required not to
remove the copyright notices from the source files. Please send any comments,
bugs, etc.\ to the author(s).



\section{How to build it}

Ebnf2ps was written in Haskell, a non--strict, purely functional 
programming language. To build Ebnf2ps you need a Haskell Compiler, 
for example the Glasgow Haskell Compilation System
\texttt{ghc}\footnote{GHC home page URL
  \texttt{http://www.dcs.glasgow.ac.uk/fp/software/ghc.html}.} or the
Haskell Compiler \texttt{hbc} by Lennart Augustsson.

\begin{enumerate}
\item  Edit the Makefile to reflect your Haskell compiler, for example
\begin{verbatim}
               HC= ghc
\end{verbatim}
\item Find out where your \TeX\ installation stores \texttt{.afm}-files (Adobe 
  font metric files). Then you can change the variable \texttt{AFMPATH} in the 
  Makefile or you can set the environment variable \texttt{AFMPATH} when running 
  the program. The \texttt{.afm}-files are absolutely necessary. If you don't 
  find them complain to your system administrator or get them yourself
  from \\ 
  \texttt{ftp://ftp.adobe.com/pub/adobe/AFMFiles/}.
\item 
  Change the variable \texttt{RGBPATH} to contain the directory where
  your X installation stores its color data base (usually
  \texttt{/usr/lib/X11/rgb.txt}). This can also be corrected later with the
  environment variable \texttt{RGBPATH}. There are fallback color definitions 
  for the 8 \emph{digital} colors (black, white, blue, green, cyan, red, magenta, 
  yellow), so nothing needs be done if that's enough for you.
\item 
  Typing \texttt{make} should produce an executable file named Ebnf2ps.
\end{enumerate}



\section{How to use it}

The syntax is
\begin{verbatim}
Ebnf2ps [options] Grammarfile Nonterminal ... [-- Nonterminal ...]
\end{verbatim}
where the \texttt{options} are described in Sec.~\ref{sec:options}, the syntax
of the input file \texttt{Grammarfile} in Sec.~\ref{sec:syntax}, and
\texttt{Nonterminal} describes the nonterminals, for which diagrams are to be
generated, as regular expressions (see Sec.~\ref{sec:regular}). The nonterminals
after the symbol \verb|--| are used only with the option \texttt{+unfold} (see 
Sec.~\ref{sec:options} and \ref{sec:unfold}).  


\subsection{Options}
\label{sec:options}

The following list shows all \textbf{options} with their \emph{parameters} and 
their (default values). All distances and sizes measured in 1/100 point (1 point 
is about 1/72 inch). Changes with respect to the last version of Ebnf2ps:
some \new{}ew features, extension of grammar transformations and bug fixes.

\begin{bfscript}{\qquad}
\item[-titleFont     \emph{font}] (Times-Roman) PostScript font used for titles 
  of diagrams, any font goes here that your printer knows and for which you have 
  fetched the AFM files. Figure \ref{fig:AFMfonts} shows a sample list of font
  names. \new
\item[-titleScale     \emph{int}] (10) Pointsize of typeface for titles of diagrams.\new
\item[-titleColor   \emph{color}] (black) Color of typeface for titles of diagrams. \new
\item[-ntFont        \emph{font}] (Times-Roman) PostScript font used for nonterminals.
\item[-ntScale        \emph{int}] (10) Pointsize of typeface for nonterminals. 
\item[-ntColor      \emph{color}] (black) Color of typeface for nonterminals.
\item[-ntBg         \emph{color}] (white) Background color of typeface for nonterminals.\new
\item[-ntBoxColor   \emph{color}] (black) Color used for outlines of boxes (nonterminals).\new
\item[-tFont         \emph{font}] (Times-Roman) PostScript font used for terminal strings. 
\item[-tScale         \emph{int}] (10) Pointsize of typeface for terminals.
\item[-tColor       \emph{color}] (black) Color of typeface for terminals.
\item[-tBg          \emph{color}] (white) Background color of typeface for terminals.\new
\item[-tBoxColor    \emph{color}] (black) Color used for outlines of boxes (terminals).\new
\item[-borderDistX    \emph{int}] (500) Horizontal distance of objects from their container.
\item[-borderDistY    \emph{int}] (500) Vertical distance of objects from their container. 
\item[-lineWidth      \emph{int}] (30) Width of lines used for connecting lines.
\item[-fatLineWidth   \emph{int}] (100) Width of lines used for drawing boxes.
\item[-lineColor    \emph{color}] (black) Color used for connecting lines.
\item[-arrowSize      \emph{int}] (200) Size of (invisible) box containing an arrow.
\item[-rgbFileName    \emph{filename}] (rgb.txt) File name for color definitions.
\item[-happy] Accept happy format input files. Happy, written by Andy Gill and Simon Marlow,
  is a parser generator system (LALR(1)) for Haskell, similar to the tool yacc for C\footnote{The
  program Happy is available by FTP from \texttt{ftp.dcs.glasgow.ac.uk} in \texttt{pub/haskell/happy}}.
\item[-yacc] Accept yacc/bison format input files.\new
\item[+ps] Produce encapsulated PostScript output (default).
\item[+fig] Produce fig output (FORMAT 3.1).
\item[+ebnf] Produce grammar in EBNF notation (file extension ".BNF") for yacc or happy 
  input format files.\new
\item[+simplify] Simplify productions (experimental, default: don't). 
\item[+unfold] Replace nonterminals in productions (experimental, default: don't). 
  See Sec.~\ref{sec:unfold}.\new
\item[-verbose] Prints some progress messages.
\item[-help] Prints a brief description of the command line, the options with their defaults,
   and the environment variables.
\end{bfscript}

\begin{figure}
  \begin{center}        
    \begin{tabular}{|l|l|l|} \hline
        AGaramond-Bold & AGaramond-BoldItalic \\
        AGaramond-Italic & AGaramond-Regular  \\
        AGaramond-Semibold & AGaramond-SemiboldItalic \\
        AGaramondAlt-Italic & AGaramondAlt-Regular \\
        AGaramondExp-Bold & AGaramondExp-BoldItalic \\
        AGaramondExp-Italic & AGaramondExp-Semibold \\
        AGaramondExp-SemiboldItalic & AvantGarde-Book \\
        AvantGarde-BookOblique & AvantGarde-Demi \\
        AvantGarde-DemiOblique & Bookman-Demi \\
        Bookman-DemiItalic & Bookman-Light \\
        Bookman-LightItalic & Courier-Bold \\
        Courier-BoldOblique & Courier-Oblique \\
        Courier & Helvetica-Bold \\
        Helvetica-BoldOblique & Helvetica-Narrow-Bold \\
        Helvetica-Narrow-BoldOblique & Helvetica-Narrow-Oblique \\
        Helvetica-Narrow & Helvetica-Oblique \\
        Helvetica & NewCenturySchlbk-Bold \\
        NewCenturySchlbk-BoldItalic & NewCenturySchlbk-Italic \\
        NewCenturySchlbk-Roman & Optima-Bold \\
        Optima &  Palatino-Bold \\
        Palatino-BoldItalic & Palatino-Italic \\
        Palatino-Oblique & Palatino-Roman \\
        StoneInformal-Bold & StoneInformal-BoldItalic \\
        StoneInformal-Italic & StoneInformal-Semibold \\
        StoneInformal-SemiboldItalic & StoneInformal \\
        StoneSans & StoneSerif-Bold \\
        StoneSerif-BoldItalic & StoneSerif-Italic \\
        StoneSerif-Semibold & StoneSerif-SemiboldItalic \\
        StoneSerif & Symbol \\
        Times-Bold & Times-BoldItalic \\
        Times-Italic & Times-Oblique \\
        Times-Roman & ZapfChancery-MediumItalic \\
        ZapfDingbats & \\ \hline                               
    \end{tabular}
  \end{center}  
  \caption{Sample list of font names}
  \label{fig:AFMfonts}
\end{figure}


\subsection{Syntax of EBNF Files}
\label{sec:syntax}

The following grammar and diagrams describe the standard input format (e.g. without 
\textbf{-yacc} or \textbf{-happy}) as well as the output format with the \textbf{+ebnf} 
option. 

\begin{verbatim}
File            = {Production};
Production      = Nonterminal [ String ] "=" Term ";" ;
Term            = Factor / "|" ;         # alternative
Factor          = ExtAtom + ;            # sequence
ExtAtom         = Atom
                | Atom "/" Atom          # repetition through Atom
                | Atom "+";              # at least one repetition
Atom            = Nonterminal
                | String                 # terminal string
                | "(" Term ")"
                | "[" Term "]"           # an optional Term
                | "{" Term "}"           # zero or more repetitions
                ;
String          = "\"" { Character } "\"" ;
Nonterminal     = letter { letter | digit | "_" | "." } ;
Character       = character 
                | ["\\"] anycharacter ;
\end{verbatim}

Or in terms of diagrams (what would you expect):
\begin{flushleft}
  \leavevmode\epsfbox{File.eps}
  \\
  \leavevmode\epsfbox{Production.eps}
  \\
  \leavevmode\epsfbox{Term.eps}
  \\
  \leavevmode\epsfbox{Factor.eps}
  \\
  \leavevmode\epsfbox{ExtAtom.eps}
  \\
  \leavevmode\epsfbox{Atom.eps}
  \\
  \leavevmode\epsfbox{String.eps}
  \\
  \leavevmode\epsfbox{Nonterminal.eps}
  \\
  \leavevmode\epsfbox{Character.eps}
\end{flushleft}

\noindent These diagrams have been produced with \texttt{Ebnf2ps
  -tFont Courier-Bold -tBg Black -tColor White -fatLineWidth 50
  ebnf.BNF '.*'}. 


\subsection{Syntax of regular expressions}
\label{sec:regular}

The regular expressions used to select nonterminal symbols have the following
syntax, which is similar to the regular expression syntax of \texttt{grep}. 

\begin{verbatim}
Regexp   = RFactor / "\\|" ["$"];
RFactor  = RExtAtom + ;
RExtAtom = RAtom ["*" | "+" | "?"] ;
RAtom    = character | "\\" character | "." | "\\(" Regexp "\\)" ;
\end{verbatim}

Regular expressions \texttt{Regexp} may be joined by the infix
operator \texttt{|}. The resulting regular expression matches any
string matching either subexpression. A subexpression may be enclosed
in parenthesis \verb|\(|, \verb|\)| to override precedence rules.
Repetition takes precedence over concatenation, which in turn takes
precedence over alternation. A regular expression matching a single
character may be followed by one of several repetition operators:

\begin{description}
\item[\texttt{?}] The preceding item is optional and matched at most once.
\item[\texttt{*}] The preceding item will be matched zero or more times.
\item[\texttt{+}] The preceding item will be matched one or more times.
\end{description}
The period \texttt{.} matches any single character, and the dollar sign 
\verb|$| match the empty string at the end of a line. Well, graphically, 
the syntax is\footnote{The diagrams have been produced with:\\
\texttt{Ebnf2ps -titleFont AvantGarde-Book -ntBoxColor White -fatLineWidth 
50 regular.BNF '.*'}}:

\begin{flushleft}
  \leavevmode\epsfbox{Regexp.eps}
  \\
  \leavevmode\epsfbox{RFactor.eps}
  \\
  \leavevmode\epsfbox{RExtAtom.eps}
  \\
  \leavevmode\epsfbox{RAtom.eps}
\end{flushleft}



\subsection{Option +unfold}
\label{sec:unfold}

The option \textbf{+unfold} allows the replacement of nonterminal
symbols in productions, through the "right side" of the production of
corresponding nonterminals. The nonterminals, which follows the symbol
\verb|--| in a command line could be specified by their full names,
or by regular expressions (see Sec.~\ref{sec:regular}). An example:
with the grammar \texttt{regular.BNF} for the regular expression
syntax in section~\ref{sec:regular} should be generated a diagram for
\texttt{RAtom}, with a replacement of nonterminal \texttt{Regexp}.
The following diagram is been produced with \texttt{Ebnf2ps +unfold
regular.BNF RAtom -- Regexp}.

\begin{flushleft}
  \leavevmode\epsfbox{RAtom_unfold.eps}    
\end{flushleft}

The following example illustrates the use of regular expressions with
the option \texttt{+unfold}. In all productions of the grammar file
\texttt{ebnf.BNF} --- which describes the syntax of EBNF files (see
Sec.~\ref{sec:syntax}) --- the nonterminal \texttt{String} and all
nonterminals that match the pattern 
\begin{verbatim}
   .*\(t\|T\)erm.*
\end{verbatim}
\noindent should be replaced. The next lines shows the command to produce 
the railroad diagrams and the progress messages of Ebnf2ps.

{\fontsize{8}{9pt}
\begin{verbatim}
% Ebnf2ps -verbose -tScale 8 -ntScale 8 +unfold ebnf.BNF '.*' -- String '.*\(t\|T\)erm.*'
using colors:
        ntColor          (0,0,0)
        tColor           (0,0,0)
        lineColor        (0,0,0)
        ntBoxColor       (0,0,0)
        tBoxColor        (0,0,0)
        ntBg             (255,255,255)
        tBg              (255,255,255)
        titleColor       (0,0,0)
from rgbPathDefault: ["/usr/lib/X11", "/usr/local/X11R5/lib/X11"]
generating nonterminals: [".*"]
from ebnf.BNF
using input path ["."]
using ebnfInput
unfold nonterminals:
        Nonterminal
        String
        Term
in the following productions
        Production
        Atom
produce layout and output
\end{verbatim}
}

\noindent Well, graphically that is:
\begin{flushleft}
  \leavevmode\epsfbox{Atom_unfold.eps}\\
  \leavevmode\epsfbox{Production_unfold.eps}
\end{flushleft}

%
\subsection{Environment variables}

The format of search pathes is a colon separated list of directories. The
default path is inserted in place of an empty directory in the list. Any file
name starting with \texttt{/} is taken as an absolute name and any file name
starting with \texttt{.} is taken relative to the current working directory.

\begin{description}
\item[AFMPATH] search path for Adobe font metric files. 
\item[EBNFINPUTS] search path for grammar input files.
\item[RGBPATH] search path for the color definitions.
\end{description}


\section{How to change it}

In order to add new options to the program you will need a modified copy of the
program \texttt{Mkhprog} with kind permission of the author Nick North 
(\texttt{ndn@seg.npl.co.uk}) which is available from URL 
\begin{verbatim}
http://www-pu.informatik.uni-tuebingen.de/users/thiemann/haskell/ebnf2ps/
\end{verbatim}

\end{document}
%%

